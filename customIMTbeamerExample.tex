\documentclass{customIMTbeamer/imt_pres}
% \documentclass[aspectratio=169]{customIMTbeamer/imt_pres}

%=========================================================================================
%================== TITLE ETC...
%=========================================================================================

\title[IMTBeamerTemplate]{IMT Beamer Template}
\subtitle{This seems to be a quite long and useless subtitle}

\author{
Alexandre Le Borgne~\inst{1}\\ \and 
David Delahaye~\inst{2}\\ \and 
Marianne Huchard~\inst{2}\\ \and 
Christelle Urtado~\inst{1}\\ \and
Sylvain Vauttier~\inst{1}}

\institute{\inst{1} LGI2P, IMT Mines Ales \& Montpellier University, Ales, France \and %
                      \inst{2} Montpellier University, CNRS, LIRMM, Montpellier, France}

\date{Date of Presentation}

\hypersetup{
pdfauthor = {Alexandre Le Borgne: https://alexandreleborgne.wp.imt.fr/},
pdfsubject = {Beamer},
pdfkeywords = {Beamer theme, imt},
pdfmoddate= {D:\pdfdate},
pdfcreator = {}
}

\begin{document}
	
	\imttitlepage
	\begin{imttoc}
		% For longer presentations use hideallsubsections option
        \begin{spacing}{.9}
		\tableofcontents%[hideallsubsections]
        \end{spacing}
%         \tableofcontents[sections={1}]
%     	\framebreak
%   		\tableofcontents[sections={1-2}]
    \end{imttoc}

\section{Introduction}

\begin{frame}{Introduction}

\begin{itemize}
  \item Your introduction goes here!
  \item Use \texttt{itemize} to organize your main points.
\end{itemize}

\vskip 1cm

\begin{block}{Examples}
Some examples of commonly used commands and features are included, to help you get started.
\end{block}

\end{frame}

\section{Some \LaTeX{} Examples}

\subsection{Tables and Figures}

\begin{frame}{Tables and Figures}

\begin{itemize}
\item Use \texttt{tabular} for basic tables --- see Table~\ref{tab:widgets}, for example.
\item You can upload a figure (JPEG, PNG or PDF) using the files menu. 
\item To include it in your document, use the \texttt{includegraphics} command (see the comment below in the source code).
\end{itemize}

% Commands to include a figure:
%\begin{figure}
%\includegraphics[width=\textwidth]{your-figure's-file-name}
%\caption{\label{fig:your-figure}Caption goes here.}
%\end{figure}

\begin{table}
\centering
\begin{tabular}{l|r}
Item & Quantity \\\hline
Widgets & 42 \\
Gadgets & 13
\end{tabular}
\caption{\label{tab:widgets}An example table.}
\end{table}

\end{frame}

\subsection{Mathematics}

\begin{frame}{Readable Mathematics}

Let $X_1, X_2, \ldots, X_n$ be a sequence of independent and identically distributed random variables with $\text{E}[X_i] = \mu$ and $\text{Var}[X_i] = \sigma^2 < \infty$, and let
$$S_n = \frac{X_1 + X_2 + \cdots + X_n}{n}
      = \frac{1}{n}\sum_{i}^{n} X_i$$
denote their mean. Then as $n$ approaches infinity, the random variables $\sqrt{n}(S_n - \mu)$ converge in distribution to a normal $\mathcal{N}(0, \sigma^2)$.

\end{frame}

\section{Blocks}\label{Blocks}


%-=-=-=-=-=-=-=-=-=-=-=-=-=-=-=-=-=-=-=-=-=-=-=-=
%	FRAME: Blocks
%-=-=-=-=-=-=-=-=-=-=-=-=-=-=-=-=-=-=-=-=-=-=-=-=

\begin{frame}{Blocks}

\begin{block}{Block Title Here}
		Great for definitions
\end{block}

\begin{alertblock}{Alert Title Here}
		Great for definitions
\end{alertblock}

\begin{exampleblock}{Example Title Here}
		Great for examples
\end{exampleblock}
\end{frame}

%-=-=-=-=-=-=-=-=-=-=-=-=-=-=-=-=-=-=-=-=-=-=-=-=
%	FRAME: Blocks
%-=-=-=-=-=-=-=-=-=-=-=-=-=-=-=-=-=-=-=-=-=-=-=-=

\begin{frame}{Blocks}

\begin{block}{Block Title Here}
	\begin{itemize}
		\item point 1
		\item point 2
	\end{itemize}
\end{block}

\begingroup
\setbeamercolor{block title}{fg=white, bg=\cnBlue}
\setbeamercolor{block body}{bg=\cnLightBlue}
\begin{block}{Blue Colored Blocks}
	Produced by using the \texttt{cblock} theme option
\end{block}
\endgroup
\end{frame}

%-=-=-=-=-=-=-=-=-=-=-=-=-=-=-=-=-=-=-=-=-=-=-=-=
%	FRAME: Additional Blocks
%-=-=-=-=-=-=-=-=-=-=-=-=-=-=-=-=-=-=-=-=-=-=-=-=

\begin{frame}{Additional Blocks}
\begin{alertblock}{Alert Block}
	Highlight important information.
\end{alertblock}

\begingroup
\setbeamercolor{block title}{fg=white, bg=\cnRed}
\setbeamercolor{block body}{bg=\cnLightRed}
\begin{block}{Red Colored Blocks}
	Produced by using the \texttt{cblock} theme option
\end{block}
\endgroup
\end{frame}

%-=-=-=-=-=-=-=-=-=-=-=-=-=-=-=-=-=-=-=-=-=-=-=-=
%	FRAME: Additional Blocks
%-=-=-=-=-=-=-=-=-=-=-=-=-=-=-=-=-=-=-=-=-=-=-=-=

\begin{frame}{Additional Blocks}

\begin{exampleblock}{Example Block}
	Examples can be good.
\end{exampleblock}

\begingroup
\setbeamercolor{block title}{fg=white, bg=\cnRed}
\setbeamercolor{block body}{bg=\cnLightRed}
\begin{block}{Red Colored Blocks}
	Produced by using the \texttt{cblock} theme option
\end{block}
\endgroup

\end{frame}

%-=-=-=-=-=-=-=-=-=-=-=-=-=-=-=-=-=-=-=-=-=-=-=-=
%	FRAME: Blocks
%-=-=-=-=-=-=-=-=-=-=-=-=-=-=-=-=-=-=-=-=-=-=-=-=

\begin{frame}{Custom Blocks}
\begingroup
\setbeamercolor{block title}{fg=white, bg=\cnDarkBlue}
\setbeamercolor{block body}{bg=\cnLightBlue}
\begin{block}{DarkBlue customization}
	Using the theme colors to generate colored blocks.
\end{block}
\endgroup

\end{frame}

\end{document}
